\documentclass[a4paper,10pt]{scrartcl}
\usepackage[utf8]{inputenc}
\usepackage{longtable}
\usepackage{amsmath}
\usepackage{amssymb}
\usepackage{textcomp}
\usepackage{listings}
\usepackage{graphicx}
\usepackage[usenames,dvipsnames,svgnames,table]{xcolor}
\usepackage{algorithm}
\usepackage{algorithmic}
\floatname{algorithm}{Algoritmo}
\renewcommand{\listalgorithmname}{Lista de algoritmos}
\renewcommand{\algorithmicrequire}{\textbf{Entrada:}}
\renewcommand{\algorithmicensure}{\textbf{Salida:}}
\renewcommand{\algorithmicend}{\textbf{fin}}
\renewcommand{\algorithmicif}{\textbf{si}}
\renewcommand{\algorithmicthen}{\textbf{entonces}}
\renewcommand{\algorithmicelse}{\textbf{en otro caso}}
\renewcommand{\algorithmicelsif}{\algorithmicelse,\ \algorithmicif}
\renewcommand{\algorithmicendif}{\algorithmicend\ \algorithmicif}
\renewcommand{\algorithmicfor}{\textbf{para }}
\renewcommand{\algorithmicforall}{\textbf{para todo}}
\renewcommand{\algorithmicdo}{\textbf{}}
\renewcommand{\algorithmicendfor}{\algorithmicend\ \algorithmicfor}
\renewcommand{\algorithmicwhile}{\textbf{mientras}}
\renewcommand{\algorithmicendwhile}{\algorithmicend\ \algorithmicwhile}
\renewcommand{\algorithmicloop}{\textbf{repetir}}
\renewcommand{\algorithmicendloop}{\algorithmicend\ \algorithmicloop}
\renewcommand{\algorithmicrepeat}{\textbf{repetir}}
\renewcommand{\algorithmicuntil}{\textbf{hasta que}}
\renewcommand{\algorithmicprint}{\textbf{imprimir}} 
\renewcommand{\algorithmicreturn}{\textbf{devolver}} 
\renewcommand{\algorithmictrue}{\textbf{true }} 
\renewcommand{\algorithmicfalse}{\textbf{false }} 
\renewcommand{\algorithmicand}{\textbf{y}} 
\renewcommand{\algorithmicor}{\textbf{o}} 

\usepackage{pstricks,pst-node,pst-tree}

\everymath{\displaystyle}
\def\O{\mathcal{O}}
\def\hora{3,6\cdot10^{9}}
\def\dia{8,64\cdot10^{10}}
\def\semana{6,048\cdot10^{11}}
\def\anio{3,15\cdot10^{13}}
\def\anios{3,15\cdot10^{16}}
\def\Problema#1#2{\textbf{Problema #1} \textsl{#2}\\}
\def\C#1{\texttt{#1}}
\def\figurename{Figura}
%opening
\title{Reto 4: Árboles}
\author{Francisco David Charte Luque\and
        Ignacio Cordón Castillo}
\date{}

\begin{document}
\maketitle
\section{Inorden no recursivo}
        \textit{Diseñar un procedimiento inorden no recursivo a imagen y semejanza
        del procedimiento preorden no recursivo que el profesor diseñó en la clase.}

 \ 
 
 \leftskip=1em
 \small
 \texttt{% Generator: GNU source-highlight, by Lorenzo Bettini, http://www.gnu.org/software/src-highlite
\noindent
\mbox{}\textcolor{ForestGreen}{void}\ \textbf{\textcolor{Black}{inordenNR}}\textcolor{BrickRed}{(}\textbf{\textcolor{Blue}{const}}\ ArbolBinario\textcolor{BrickRed}{\textless{}}\textcolor{ForestGreen}{int}\textcolor{BrickRed}{\textgreater{}\&}\ a\textcolor{BrickRed}{)}\textcolor{Red}{\{} \\
\mbox{}\ \ \ \ ArbolBinario\textcolor{BrickRed}{\textless{}}\textcolor{ForestGreen}{int}\textcolor{BrickRed}{\textgreater{}::}\textcolor{TealBlue}{Nodo}\ actual\textcolor{BrickRed}{;} \\
\mbox{}\ \ \ \ \textcolor{TealBlue}{stack\textless{}ArbolBinario\textless{}int\textgreater{}::Nodo\textgreater{}}\ p\textcolor{BrickRed}{;} \\
\mbox{}\ \ \ \ \textcolor{ForestGreen}{bool}\ subiendo\ \textcolor{BrickRed}{=}\ \textbf{\textcolor{Blue}{false}}\textcolor{BrickRed}{;}\ \ \ \  \\
\mbox{} \\
\mbox{}\ \ \ \ actual\textcolor{BrickRed}{=}a\textcolor{BrickRed}{.}\textbf{\textcolor{Black}{raiz}}\textcolor{BrickRed}{();} \\
\mbox{}\ \ \ \ p\textcolor{BrickRed}{.}\textbf{\textcolor{Black}{push}}\textcolor{BrickRed}{(}ArbolBinario\textcolor{BrickRed}{\textless{}}\textcolor{ForestGreen}{int}\textcolor{BrickRed}{\textgreater{}::}nodo$\_$nulo\textcolor{BrickRed}{);} \\
\mbox{}\ \ \ \ p\textcolor{BrickRed}{.}\textbf{\textcolor{Black}{push}}\textcolor{BrickRed}{(}actual\textcolor{BrickRed}{);} \\
\mbox{}\ \ \ \  \\
\mbox{}\ \ \ \ \textbf{\textcolor{Blue}{while}}\ \textcolor{BrickRed}{(}actual\ \textcolor{BrickRed}{!=}\ ArbolBinario\textcolor{BrickRed}{\textless{}}\textcolor{ForestGreen}{int}\textcolor{BrickRed}{\textgreater{}::}nodo$\_$nulo\textcolor{BrickRed}{)}\textcolor{Red}{\{} \\
\mbox{}\ \ \ \ \ \ \ \ \textbf{\textcolor{Blue}{if}}\ \textcolor{BrickRed}{(}a\textcolor{BrickRed}{.}\textbf{\textcolor{Black}{izquierda}}\textcolor{BrickRed}{(}actual\textcolor{BrickRed}{)}\ \textcolor{BrickRed}{!=}\ ArbolBinario\textcolor{BrickRed}{\textless{}}\textcolor{ForestGreen}{int}\textcolor{BrickRed}{\textgreater{}::}nodo$\_$nulo\ \textcolor{BrickRed}{\&\&}\ \textcolor{BrickRed}{!}subiendo\textcolor{BrickRed}{)}\textcolor{Red}{\{} \\
\mbox{}\ \ \ \ \ \ \ \ \ \ \ \ \textit{\textcolor{Brown}{//\ Pasamos\ a\ manejar\ el\ hijo\ izquierdo}} \\
\mbox{}\ \ \ \ \ \ \ \ \ \ \ \ actual\ \textcolor{BrickRed}{=}\ a\textcolor{BrickRed}{.}\textbf{\textcolor{Black}{izquierda}}\textcolor{BrickRed}{(}actual\textcolor{BrickRed}{);} \\
\mbox{}\ \ \ \ \ \ \ \ \ \ \ \ p\textcolor{BrickRed}{.}\textbf{\textcolor{Black}{push}}\textcolor{BrickRed}{(}actual\textcolor{BrickRed}{);} \\
\mbox{}\ \ \ \ \ \ \ \ \textcolor{Red}{\}}\ \textbf{\textcolor{Blue}{else}}\ \textcolor{Red}{\{} \\
\mbox{}\ \ \ \ \ \ \ \ \ \ \ \ cout\ \textcolor{BrickRed}{\textless{}\textless{}}\ a\textcolor{BrickRed}{.}\textbf{\textcolor{Black}{etiqueta}}\textcolor{BrickRed}{(}actual\textcolor{BrickRed}{)}\ \textcolor{BrickRed}{\textless{}\textless{}}\ \texttt{\textcolor{Red}{'\ '}}\textcolor{BrickRed}{;} \\
\mbox{}\ \ \ \ \ \ \ \ \ \ \ \ p\textcolor{BrickRed}{.}\textbf{\textcolor{Black}{pop}}\textcolor{BrickRed}{();} \\
\mbox{}\ \ \ \ \ \ \ \ \ \ \ \ subiendo\ \textcolor{BrickRed}{=}\ \textbf{\textcolor{Blue}{true}}\textcolor{BrickRed}{;} \\
\mbox{}\ \ \ \ \ \ \ \ \ \ \ \  \\
\mbox{}\ \ \ \ \ \ \ \ \ \ \ \ \textbf{\textcolor{Blue}{if}}\ \textcolor{BrickRed}{(}a\textcolor{BrickRed}{.}\textbf{\textcolor{Black}{derecha}}\textcolor{BrickRed}{(}actual\textcolor{BrickRed}{)}\ \textcolor{BrickRed}{!=}\ ArbolBinario\textcolor{BrickRed}{\textless{}}\textcolor{ForestGreen}{int}\textcolor{BrickRed}{\textgreater{}::}nodo$\_$nulo\textcolor{BrickRed}{)}\ \textcolor{Red}{\{} \\
\mbox{}\ \ \ \ \ \ \ \ \ \ \ \ \ \ \ \ \textit{\textcolor{Brown}{//\ Pasamos\ a\ manejar\ el\ hijo\ derecho}} \\
\mbox{}\ \ \ \ \ \ \ \ \ \ \ \ \ \ \ \ actual\ \textcolor{BrickRed}{=}\ a\textcolor{BrickRed}{.}\textbf{\textcolor{Black}{derecha}}\textcolor{BrickRed}{(}actual\textcolor{BrickRed}{);} \\
\mbox{}\ \ \ \ \ \ \ \ \ \ \ \ \ \ \ \ p\textcolor{BrickRed}{.}\textbf{\textcolor{Black}{push}}\textcolor{BrickRed}{(}actual\textcolor{BrickRed}{);} \\
\mbox{}\ \ \ \ \ \ \ \ \ \ \ \ \ \ \ \ subiendo\ \textcolor{BrickRed}{=}\ \textbf{\textcolor{Blue}{false}}\textcolor{BrickRed}{;} \\
\mbox{}\ \ \ \ \ \ \ \ \ \ \ \ \textcolor{Red}{\}}\ \textbf{\textcolor{Blue}{else}}\ \textcolor{Red}{\{}\  \\
\mbox{}\ \ \ \ \ \ \ \ \ \ \ \ \ \ \ \ \textit{\textcolor{Brown}{//\ Trataremos\ de\ saltar\ al\ hermano}} \\
\mbox{}\ \ \ \ \ \ \ \ \ \ \ \ \ \ \ \ actual\ \textcolor{BrickRed}{=}\ p\textcolor{BrickRed}{.}\textbf{\textcolor{Black}{top}}\textcolor{BrickRed}{();} \\
\mbox{}\ \ \ \ \ \ \ \ \ \ \ \ \textcolor{Red}{\}} \\
\mbox{}\ \ \ \ \ \ \ \ \textcolor{Red}{\}} \\
\mbox{}\ \ \ \ \textcolor{Red}{\}} \\
\mbox{}\textcolor{Red}{\}}
}
 \normalsize
 
 \leftskip=0em
 \section{Codificación de árbol binario}
         \textit{Dar un procedimiento para guardar un árbol binario en disco de forma que se
         recupere la estructura jerárquica de forma unívoca usando el mínimo número
         de centinelas que veais posible.}
         
  Se propone lo siguiente:
 
 Deseamos conocer para cada nodo del árbol si tiene dos hijos, sólo el
 izquierdo, solo el derecho o ninguno. Se empleará el preorden del árbol
 binario, haciendo que a cada nodo le preceda, caso de ser necesario, uno
 de los tres centinelas siguientes:
  
 \begin{enumerate}
 \item[i.] \texttt{<} Si le falta el hijo izquierdo
 \item[ii.] \texttt{>} Si le falta el hijo derecho
 \item[iii.] \texttt{-} Si no tiene hijos
 \end{enumerate}
 
 No se hará empleo de ningún centinela si el nodo tiene ambos hijos.
   
      \begin{algorithm}[H]
      \begin{algorithmic}[1]
     \REQUIRE \ \\
         \texttt{bin\_tree}, árbol binario leído codificado\\\
     
     \STATE{Crear pila \texttt{nodos}, 
            pila \texttt{hijos}}
     \STATE{\texttt{centinelas = \{-,<,>\}}}
     \\\
     \FORALL{\texttt{elemento} $en$ \texttt{bin\_tree}}
       \IF{\texttt{elemento} $\in$ \texttt{centinelas}}
         \STATE{Apilar \texttt{elemento} en \texttt{hijos}}
         \STATE{\texttt{anterior} $\gets$ \texttt{actual}}
       \ELSE
         \STATE{Apilar \texttt{elemento} en \texttt{nodos}}
         \IF{\texttt{anterior} $\notin$ \texttt{centinelas}}
           \STATE{Apilar * en \texttt{hijos}}
         \ENDIF
       \ENDIF
     \ENDFOR
     \\\
     \STATE{Crear arbol \texttt{nuevo}}
     \STATE{\texttt{raiz} $\gets$ raíz de \texttt{nuevo}}
     \STATE{Llamar al algoritmo \ref{rec2} con parámetros \texttt{raiz}, \texttt{nodos}, \texttt{hijos}}
     \\\
     \RETURN \texttt{nuevo}
      \end{algorithmic}
      \caption{Recuperado del árbol (I)}
      \label{rec1}
      \end{algorithm}
   
      \begin{algorithm}[H]
      \begin{algorithmic}[1]
     \REQUIRE \ \\
         \texttt{actual}, nodo sobre el que añadir descendientes\\
         \texttt{nodos}, pila de nodos\\
         \texttt{hijos}, pila de centinelas\\\  
     \IF{Tope \texttt{hijos} $\neq$ \texttt{-} \AND
         Tope \texttt{hijos} $\neq$ \texttt{<}}
         \STATE{Hijo izquierda de \texttt{actual}
                $\gets$ tope de \texttt{hijos}}
         \STATE{Sacar el tope de \texttt{hijos}}
         \STATE{Sacar el tope de \texttt{nodos}}
         \STATE{Llamar al algoritmo \ref{rec2} con parámetros
               \texttt{hijo izquierda de actual}, \texttt{nodos}, \texttt{hijos}}
     \ELSE
       \STATE{Hijo izquierda de \texttt{actual} $\gets \emptyset$}
     \ENDIF
     \\\
     \IF{Tope \texttt{hijos} $\neq$ \texttt{-} \AND
         Tope \texttt{hijos} $\neq$\texttt{>}}
       \STATE{Hijo derecha de \texttt{actual}
               $\gets$ tope de \texttt{nodos}}
        \STATE{Sacar el tope de \texttt{hijos}}
        \STATE{Sacar el tope de \texttt{nodos}}
       \STATE{Llamar al algoritmo \ref{rec2} con parámetros 
             \texttt{hijo derecha de actual}, \texttt{nodos}, \texttt{hijos}}
     \ELSE
         \STATE{Hijo derecha de \texttt{actual} $\gets \emptyset$}
     \ENDIF
      \end{algorithmic}
      \caption{Recuperado del árbol (II)}
      \label{rec2}
      \end{algorithm}
\end{document}
