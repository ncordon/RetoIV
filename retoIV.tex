\documentclass[a4paper,10pt]{scrartcl}
\usepackage[utf8]{inputenc}
\usepackage{longtable}
\usepackage{amsmath}
\usepackage{amssymb}
\usepackage{textcomp}
\usepackage{listings}
\usepackage{graphicx}
\usepackage[usenames,dvipsnames,svgnames,table]{xcolor}

\everymath{\displaystyle}
\def\O{\mathcal{O}}
\def\hora{3,6\cdot10^{9}}
\def\dia{8,64\cdot10^{10}}
\def\semana{6,048\cdot10^{11}}
\def\anio{3,15\cdot10^{13}}
\def\anios{3,15\cdot10^{16}}
\def\Problema#1#2{\textbf{Problema #1} \textsl{#2}\\}
\def\C#1{\texttt{#1}}
\def\figurename{Figura}
%opening
\title{Reto 4: Árboles}
\author{Francisco David Charte Luque\and
        Ignacio Cordón Castillo}
\date{}

\begin{document}
\maketitle
\begin{enumerate}

 \item Diseñar un procedimiento inorden no recursivo a imagen y semejanza
 del procedimiento preorden no recursivo que el profesor diseñó en la clase:
 \leftskip=1cm
 % Generator: GNU source-highlight, by Lorenzo Bettini, http://www.gnu.org/software/src-highlite
\noindent
\mbox{}\textcolor{ForestGreen}{void}\ \textbf{\textcolor{Black}{inordenNR}}\textcolor{BrickRed}{(}\textbf{\textcolor{Blue}{const}}\ ArbolBinario\textcolor{BrickRed}{\textless{}}\textcolor{ForestGreen}{int}\textcolor{BrickRed}{\textgreater{}\&}\ a\textcolor{BrickRed}{)}\textcolor{Red}{\{} \\
\mbox{}\ \ \ \ ArbolBinario\textcolor{BrickRed}{\textless{}}\textcolor{ForestGreen}{int}\textcolor{BrickRed}{\textgreater{}::}\textcolor{TealBlue}{Nodo}\ actual\textcolor{BrickRed}{;} \\
\mbox{}\ \ \ \ \textcolor{TealBlue}{stack\textless{}ArbolBinario\textless{}int\textgreater{}::Nodo\textgreater{}}\ p\textcolor{BrickRed}{;} \\
\mbox{}\ \ \ \ \textcolor{ForestGreen}{bool}\ subiendo\ \textcolor{BrickRed}{=}\ \textbf{\textcolor{Blue}{false}}\textcolor{BrickRed}{;}\ \ \ \  \\
\mbox{} \\
\mbox{}\ \ \ \ actual\textcolor{BrickRed}{=}a\textcolor{BrickRed}{.}\textbf{\textcolor{Black}{raiz}}\textcolor{BrickRed}{();} \\
\mbox{}\ \ \ \ p\textcolor{BrickRed}{.}\textbf{\textcolor{Black}{push}}\textcolor{BrickRed}{(}ArbolBinario\textcolor{BrickRed}{\textless{}}\textcolor{ForestGreen}{int}\textcolor{BrickRed}{\textgreater{}::}nodo$\_$nulo\textcolor{BrickRed}{);} \\
\mbox{}\ \ \ \ p\textcolor{BrickRed}{.}\textbf{\textcolor{Black}{push}}\textcolor{BrickRed}{(}actual\textcolor{BrickRed}{);} \\
\mbox{}\ \ \ \  \\
\mbox{}\ \ \ \ \textbf{\textcolor{Blue}{while}}\ \textcolor{BrickRed}{(}actual\ \textcolor{BrickRed}{!=}\ ArbolBinario\textcolor{BrickRed}{\textless{}}\textcolor{ForestGreen}{int}\textcolor{BrickRed}{\textgreater{}::}nodo$\_$nulo\textcolor{BrickRed}{)}\textcolor{Red}{\{} \\
\mbox{}\ \ \ \ \ \ \ \ \textbf{\textcolor{Blue}{if}}\ \textcolor{BrickRed}{(}a\textcolor{BrickRed}{.}\textbf{\textcolor{Black}{izquierda}}\textcolor{BrickRed}{(}actual\textcolor{BrickRed}{)}\ \textcolor{BrickRed}{!=}\ ArbolBinario\textcolor{BrickRed}{\textless{}}\textcolor{ForestGreen}{int}\textcolor{BrickRed}{\textgreater{}::}nodo$\_$nulo\ \textcolor{BrickRed}{\&\&}\ \textcolor{BrickRed}{!}subiendo\textcolor{BrickRed}{)}\textcolor{Red}{\{} \\
\mbox{}\ \ \ \ \ \ \ \ \ \ \ \ \textit{\textcolor{Brown}{//\ Pasamos\ a\ manejar\ el\ hijo\ izquierdo}} \\
\mbox{}\ \ \ \ \ \ \ \ \ \ \ \ actual\ \textcolor{BrickRed}{=}\ a\textcolor{BrickRed}{.}\textbf{\textcolor{Black}{izquierda}}\textcolor{BrickRed}{(}actual\textcolor{BrickRed}{);} \\
\mbox{}\ \ \ \ \ \ \ \ \ \ \ \ p\textcolor{BrickRed}{.}\textbf{\textcolor{Black}{push}}\textcolor{BrickRed}{(}actual\textcolor{BrickRed}{);} \\
\mbox{}\ \ \ \ \ \ \ \ \textcolor{Red}{\}}\ \textbf{\textcolor{Blue}{else}}\ \textcolor{Red}{\{} \\
\mbox{}\ \ \ \ \ \ \ \ \ \ \ \ cout\ \textcolor{BrickRed}{\textless{}\textless{}}\ a\textcolor{BrickRed}{.}\textbf{\textcolor{Black}{etiqueta}}\textcolor{BrickRed}{(}actual\textcolor{BrickRed}{)}\ \textcolor{BrickRed}{\textless{}\textless{}}\ \texttt{\textcolor{Red}{'\ '}}\textcolor{BrickRed}{;} \\
\mbox{}\ \ \ \ \ \ \ \ \ \ \ \ p\textcolor{BrickRed}{.}\textbf{\textcolor{Black}{pop}}\textcolor{BrickRed}{();} \\
\mbox{}\ \ \ \ \ \ \ \ \ \ \ \ subiendo\ \textcolor{BrickRed}{=}\ \textbf{\textcolor{Blue}{true}}\textcolor{BrickRed}{;} \\
\mbox{}\ \ \ \ \ \ \ \ \ \ \ \  \\
\mbox{}\ \ \ \ \ \ \ \ \ \ \ \ \textbf{\textcolor{Blue}{if}}\ \textcolor{BrickRed}{(}a\textcolor{BrickRed}{.}\textbf{\textcolor{Black}{derecha}}\textcolor{BrickRed}{(}actual\textcolor{BrickRed}{)}\ \textcolor{BrickRed}{!=}\ ArbolBinario\textcolor{BrickRed}{\textless{}}\textcolor{ForestGreen}{int}\textcolor{BrickRed}{\textgreater{}::}nodo$\_$nulo\textcolor{BrickRed}{)}\ \textcolor{Red}{\{} \\
\mbox{}\ \ \ \ \ \ \ \ \ \ \ \ \ \ \ \ \textit{\textcolor{Brown}{//\ Pasamos\ a\ manejar\ el\ hijo\ derecho}} \\
\mbox{}\ \ \ \ \ \ \ \ \ \ \ \ \ \ \ \ actual\ \textcolor{BrickRed}{=}\ a\textcolor{BrickRed}{.}\textbf{\textcolor{Black}{derecha}}\textcolor{BrickRed}{(}actual\textcolor{BrickRed}{);} \\
\mbox{}\ \ \ \ \ \ \ \ \ \ \ \ \ \ \ \ p\textcolor{BrickRed}{.}\textbf{\textcolor{Black}{push}}\textcolor{BrickRed}{(}actual\textcolor{BrickRed}{);} \\
\mbox{}\ \ \ \ \ \ \ \ \ \ \ \ \ \ \ \ subiendo\ \textcolor{BrickRed}{=}\ \textbf{\textcolor{Blue}{false}}\textcolor{BrickRed}{;} \\
\mbox{}\ \ \ \ \ \ \ \ \ \ \ \ \textcolor{Red}{\}}\ \textbf{\textcolor{Blue}{else}}\ \textcolor{Red}{\{}\  \\
\mbox{}\ \ \ \ \ \ \ \ \ \ \ \ \ \ \ \ \textit{\textcolor{Brown}{//\ Trataremos\ de\ saltar\ al\ hermano}} \\
\mbox{}\ \ \ \ \ \ \ \ \ \ \ \ \ \ \ \ actual\ \textcolor{BrickRed}{=}\ p\textcolor{BrickRed}{.}\textbf{\textcolor{Black}{top}}\textcolor{BrickRed}{();} \\
\mbox{}\ \ \ \ \ \ \ \ \ \ \ \ \textcolor{Red}{\}} \\
\mbox{}\ \ \ \ \ \ \ \ \textcolor{Red}{\}} \\
\mbox{}\ \ \ \ \textcolor{Red}{\}} \\
\mbox{}\textcolor{Red}{\}}

 \item Dar un procedimiento para guardar un árbol binario en disco de forma
 que se recupere la estructura jerárquica de forma unívoca usando el mínimo
 número de centinelas que veais posible.
 
 
\end{enumerate}
\end{document}
