\documentclass[a4paper,10pt]{scrartcl}
\usepackage[utf8]{inputenc}
\usepackage{longtable}
\usepackage{amsmath}
\usepackage{amssymb}
\usepackage{textcomp}
\usepackage{listings}
\usepackage{graphicx}
\hyphenation{re-fe-ren-cia}

\everymath{\displaystyle}
\def\O{\mathcal{O}}
\def\hora{3,6\cdot10^{9}}
\def\dia{8,64\cdot10^{10}}
\def\semana{6,048\cdot10^{11}}
\def\anio{3,15\cdot10^{13}}
\def\anios{3,15\cdot10^{16}}
\def\Problema#1#2{\textbf{Problema #1} \textsl{#2}\\}
\def\C#1{\texttt{#1}}
\def\figurename{Figura}
%opening
\title{Reto 1: Eficiencia}
\author{Francisco David Charte Luque\and
        Ignacio Cordón Castillo}
\date{}

\begin{document}
\maketitle
\begin{enumerate}

 \item Diseñar un procedimiento inorden no recursivo a imagen y semejanza
 del procedimiento preorden no recursivo que el profesor diseñó en la clase:
 
 \leftskip=1cm
 \begin{verbatim}
 
 \end{verbatim}
\end{enumerate}
\end{document}
